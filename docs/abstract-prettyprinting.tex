\documentclass[11pt]{article}
\usepackage{palatino}
\usepackage{amsmath}
\usepackage{pdflscape}
\usepackage[margin=2cm]{geometry}
\begin{document}

\title{Prettyprinting Intermediate Computations from a Bytecode}
\maketitle

\noindent Reference: \textit{Functional programming languages Part II: abstract machines, Xavier Leroy, INRIA Rocquencourt, MPRI 2-4-2, 2007}

\section{Programs}

Programs are defined like this. Variable accesses have been converted to deBruijn indices when the program was converted from an OCaml one.

{\small\begin{verbatim}
type op = Add | Sub | Mul | Div

type prog =
  Int of int
| Bool of bool
| Var of int
| Eq of prog * prog
| Op of prog * op * prog
| Apply of prog * prog
| Lambda of prog
| Let of prog * prog
| If of prog * prog * prog\end{verbatim}}

\noindent For example, the OCaml program

\medskip
\noindent\texttt{let x = 5 in if x = 4 then 1 else (fun x -> x + 1) 2}
\medskip

\noindent may be represented as:

\medskip
{\small
\begin{verbatim}Let(Int 5,
    If(Eq(Var 1, Int 4),
       Int 1,
       Apply(Lambda(Op(Var 1, Add, Int 1), Int 2))))\end{verbatim}}


\section{Compilation Scheme}

The abstract machine instructions are as followed (Leroy plus BOOL, IF, EQ)

\medskip

\noindent
EMPTY\\
INT(integer)\\
BOOL(boolean)\\
OP(op)\\
EQ\\
ACCESS(integer)\\
CLOSURE(instructions)\\
LET\\
ENDLET\\
APPLY\\
RETURN\\
IF
\medskip

\noindent Here is the compilation scheme, again extended from Leroy:

\begin{align*}
\mathcal{C}(\texttt{Int}(i)) &= \text{INT}(i)\\
\mathcal{C}(\texttt{Bool}(b)) &= \text{BOOL}(b)\\
\mathcal{C}(\texttt{Op}(a, \oplus, b)) &= \mathcal{C}(a); \mathcal{C}(b); \text{OP}(\oplus)\\
\mathcal{C}(\texttt{Eq}(a, b)) &= \mathcal{C}(a); \mathcal{C}(b); \text{EQ}\\
\mathcal{C}(\texttt{Var}(n)) &= \text{ACCESS}(n)\\
\mathcal{C}(\texttt{Lambda}(a)) &= \text{CLOSURE}(\mathcal{C}(a); \text{RETURN})\\
\mathcal{C}(\texttt{Let}(a, b)) &= \mathcal{C}(a); \text{LET}; \mathcal{C}(b); \text{ENDLET}\\
\mathcal{C}(\texttt{Apply}(a, b)) &= \mathcal{C}(a); \mathcal{C}(b); \text{APPLY}\\
\mathcal{C}(\texttt{If}(a, b, c)) &= \mathcal{C}(\texttt{Lambda}(b)); \mathcal{C}(\texttt{Lambda}(c)); \mathcal{C}(a); \text{IF}
\end{align*}

\noindent So our example


\medskip
\noindent\texttt{let x = 5 in if x = 4 then 1 else (fun x -> x + 1) 2}
\medskip


\noindent compiles to (including an EMPTY at the end):

\medskip

\noindent INT 5\\
LET\\
CLOSURE\\
\phantom{\ \ }\ INT 1\\
\phantom{\ \ }\ RETURN\\
CLOSURE\\
\phantom{\ \ } CLOSURE\\
\phantom{\ \ \ \ } ACCESS 1\\
\phantom{\ \ \ \ } INT 1\\
\phantom{\ \ \ \ } OP +\\
\phantom{\ \ \ \ } RETURN\\
\phantom{\ \ } INT 2\\
\phantom{\ \ } APPLY\\
\phantom{\ \ } RETURN\\
ACCESS 1\\
INT 4\\
EQ\\
BRANCH\\
ENDLET\\
EMPTY

\medskip

\section{Evaluation Scheme}

Here is the evaluation scheme $\mathcal{E}$, again extended from Leroy.

\bigskip

\begin{tabular}{l|l|l||l|l|l}
\multicolumn{3}{c}{Machine state before}&\multicolumn{3}{c}{Machine state after}                       \\
Code                    & Env   & Stack                          & Code   & Env    & Stack             \\
$\text{INT}(i);c$       & $e$   & $s$                            & $c$    & $e$    & $i.s$             \\
$\text{BOOL}(b);c$      & $e$   & $s$                            & $c$    & $e$    & $b.s$             \\
$\text{OP}(\oplus);c$   & $e$   & $i.i'.s$                       & $c$    & $e$    & $\oplus(i, i').s$ \\
$\text{EQ};c$           & $e$   & $i.i'.s$                       & $c$    & $e$    & $(i = i').s$      \\
$\text{ACCESS}(n);c$    & $e$   & $s$                            & $c$    & $e$    & $e(n).s$          \\
$\text{CLOSURE}(c');c$  & $e$   & $s$                            & $c$    & $e$    & $c'[e].s$         \\
$\text{LET};c$          & $e$   & $v.s$                          & $c$    & $v.e$  & $s$               \\
$\text{ENDLET};c$       & $v.e$ & $s$                            & $c$    & $e$    & $s$               \\
$\text{APPLY;c}$        & $e$   & $v.c'[e'].s$                   & $c'$   & $v.e'$ & $c.e.s$           \\
$\text{RETURN;c}$       & $e$   & $v.c'.e'.s$                    & $c'$   & $e'$   & $v.s$             \\
$\text{IF;c}$           & $e$   & $\textsf{T}.c'[e'].c''[e''].s$ & $c'$   & $e'$   & $c[e].s$          \\
$\text{IF;c}$           & $e$   & $\textsf{F}.c'[e'].c''[e''].s$ & $c''$  & $e''$  & $c[e].s$          \\
\end{tabular}

\bigskip

\noindent The final result is at the top of the stack when the code is EMPTY.

\section{Decompilation Scheme}

We need to be able to decompile:

\begin{itemize}
\item Any program which has been compiled by the compilation scheme above.
\item Certain incomplete evaluations under the evaluation scheme above. That is to say, given $(c, s)$ we can decompile a program which represents the evaluation at that stage. We need not be able to decompile arbitrary $(c, e, s)$ triples.
\end{itemize}

\noindent  We add names to \texttt{VarAccess}, \texttt{Lambda} and \texttt{Let}:

\begin{verbatim}
type prog =
  Int of int
| Bool of bool
| VarAccess of name * int
| Eq of prog * prog
| Op of prog * op * prog
| Apply of prog * prog
| Lambda of name * prog
| Let of name * prog * prog
| If of prog * prog * prog\end{verbatim}

\noindent Similarly, we add names to the ACCESS, CLOSURE and LET instructions (not required for evaluation, but only for decompilation).

\bigskip

\noindent
EMPTY\\
INT(integer)\\
BOOL(boolean)\\\
OP(op)\\
EQ\\
ACCESS(name, integer)\\
CLOSURE(name, instructions)\\
LET(name)\\
ENDLET\\
APPLY\\
RETURN\\
IF
\medskip

\noindent Decompilation is performed by going through the instructions in order, holding a stack a little like the evaluation stack, but which may also contain decompiled program fragments -- the empty stack is written $\{\}$. When we have gone through all the instructions, the final program is at the top of the stack. We do not need the environment, since we are not running the code, just decompiling it.

\begin{align*}
\mathcal{D}(\text{EMPTY}, v.s) &= v\\
\mathcal{D}(\text{INT}(i); c, s) &= \mathcal{D}(c, \texttt{Int}(i).s)\\
\mathcal{D}(\text{BOOL}(i); c, s) &= \mathcal{D}(c, \texttt{Bool}(b).s)\\
\mathcal{D}(\text{OP}(\oplus); c, i.i'.s) &= \mathcal{D}(c, \texttt{Op}(i, \oplus, i').s) \\
\mathcal{D}(\text{EQ}; c, i.i'.s) &= \mathcal{D}(c, \texttt{Eq}(i, i').s)\\
\mathcal{D}(\text{ACCESS}(n, l); c, s) &= \mathcal{D}(c, \texttt{VarAccess}(n, l).s)\\
\mathcal{D}(\text{CLOSURE}(n, c'); c, s) &= \mathcal{D}(c, c'[n, \{\}].s)\\
\mathcal{D}(\text{LET}(n); c, v.s) &= \texttt{Let}(n, v, \mathcal{D}(c, s))\\
\mathcal{D}(\text{ENDLET}; c, s) &= \mathcal{D}(c, s)\\
\mathcal{D}(\text{APPLY}; c, v.c'[n, e'].s) &= \texttt{Apply}(\texttt{Lambda(n}, \mathcal{D}(c', \{\}\texttt{))}, v)\\
\mathcal{D}(\text{RETURN}; c, v.c'.e'.s) &= \mathcal{D}(c', v.s)\\
\mathcal{D}(\text{RETURN}; c, s) &= \mathcal{D}(c, s)\\
\mathcal{D}(\text{IF}; c, e.c'[e'].c''[e''].s) &= \mathcal{D}(c, \texttt{If}(e, \mathcal{D}(c', s), \mathcal{D}(c'', s)).s)
\end{align*}

\noindent This decompiler works for:

\begin{itemize}
\item Any program-stack pair (P, \{\}) where P was compiled by $\mathcal{C}$ above.
\item program,stack pair (P, S) which is an intermediate state of the evaluation procedure $\mathcal{E}$ (minus the environment) where P begins with OP or APPLY.
\end{itemize}

\noindent Our example program decompiles properly from bytecode.

\section{Worked examples}

The following pages contain a worked example of the compilation $\mathcal{C}$, the evaluation $\mathcal{E}$, and full-program and partial-evaluation invocations of the decompiler $\mathcal{D}$.


\begin{landscape}

\noindent Compilation under $\mathcal{C}$:

\bigskip

{\small

\noindent$\mathcal{C}$(\texttt{Let(Int 5, If(Eq(Var 1, Int 4), Int 1, Apply(Lambda(Op(Var 1, Add, Int 1), Int 2))))})

\smallskip
\noindent Rule $\mathcal{C}$-\texttt{Let}
\smallskip

\noindent $\mathcal{C}$(\texttt{Int 5}); LET; $\mathcal{C}$(\texttt{If(Eq(Var 1, Int 4), Int 1, Apply(Lambda(Op(Var 1, Add, Int 1), Int 2)))}); ENDLET

\smallskip
\noindent Rule $\mathcal{C}$-\texttt{Int}
\smallskip

\noindent INT 5; LET; $\mathcal{C}$(\texttt{If(Eq(Var 1, Int 4), Int 1, Apply(Lambda(Op(Var 1, Add, Int 1), Int 2)))}); ENDLET

\smallskip
\noindent Rule $\mathcal{C}$-\texttt{If}
\smallskip

\noindent INT 5; LET; $\mathcal{C}$(\texttt{Lambda (Int 1)}); $\mathcal{C}$(\texttt{Lambda(Apply(Lambda(Op(Var 1, Add, Int 1), Int 2)))}); $\mathcal{C}$(\texttt{Eq(Var 1, Int 4)}); IF; ENDLET


\smallskip
\noindent Rule $\mathcal{C}$-\texttt{Eq} then Rule $\mathcal{C}$-\texttt{Eq} then Rule $\mathcal{C}$-\texttt{Eq}
\smallskip

\noindent INT 5; LET; $\mathcal{C}$(\texttt{Lambda (Int 1)}); $\mathcal{C}$(\texttt{Lambda(Apply(Lambda(Op(Var 1, Add, Int 1), Int 2)))}); ACCESS 1; INT 4; EQ; IF; ENDLET

\smallskip
\noindent Rule $\mathcal{C}$-\texttt{Lambda} then Rule $\mathcal{C}$-\texttt{Int}
\smallskip

\noindent INT 5; LET; CLOSURE [INT 1; RETURN]; $\mathcal{C}$(\texttt{Lambda(Apply(Lambda(Op(Var 1, Add, Int 1), Int 2)))}); ACCESS 1; INT 4; EQ; IF; ENDLET

\smallskip
\noindent Rule $\mathcal{C}$-\texttt{Lambda}
\smallskip

\noindent INT 5; LET; CLOSURE [INT 1; RETURN]; CLOSURE [$\mathcal{C}$(\texttt{Apply(Lambda(Op(Var 1, Add, Int 1), Int 2))}); RETURN]; ACCESS 1; INT 4; EQ; IF; ENDLET

\smallskip
\noindent Rule $\mathcal{C}$-\texttt{Apply}
\smallskip

\noindent INT 5; LET; CLOSURE [INT 1; RETURN]; CLOSURE [$\mathcal{C}$(\texttt{Lambda(Op(Var 1, Add, Int 1))}; $\mathcal{C}$(\texttt{Int 2}); APPLY; RETURN]; ACCESS 1; INT 4; EQ; IF; ENDLET


\smallskip
\noindent Rule $\mathcal{C}$-\texttt{Int} then Rule $\mathcal{C}$-\texttt{Lambda} then Rule $\mathcal{C}$-\texttt{Op} then Rule $\mathcal{C}$-\texttt{Var} then Rule $\mathcal{C}$-\texttt{Int}
\smallskip

\noindent INT 5; LET; CLOSURE [INT 1; RETURN]; CLOSURE [CLOSURE [ACCESS 1; INT 1; OP +; RETURN]; INT 2; APPLY; RETURN]; ACCESS 1; INT 4; EQ; IF; ENDLET


}

\newpage

\noindent Evaluation under $\mathcal{E}$. Stacks and environments are written \{items\}, and a closure on the stack is written [instructions]\{environment\}. Environments may be put on the stack.

\bigskip

{\small
\begin{tabular}{l||p{9.5cm}|l|p{9.5cm}}
& \multicolumn{3}{c}{Machine state after}                       \\
Instruction & Code      & Env   & Stack\\
   -        & INT 5; LET; CLOSURE [INT 1; RETURN]; CLOSURE [CLOSURE [ACCESS 1; INT 1; OP +; RETURN]; INT 2; APPLY; RETURN]; ACCESS 1; INT 4; EQ; IF; ENDLET & \{\}  & \{\}\\

INT & LET; CLOSURE [INT 1; RETURN]; CLOSURE [CLOSURE [ACCESS 1; INT 1; OP +; RETURN]; INT 2; APPLY; RETURN]; ACCESS 1; INT 4; EQ; IF; ENDLET& \{\}   & \{5\}\\

LET & CLOSURE [INT 1; RETURN]; CLOSURE [CLOSURE [ACCESS 1; INT 1; OP +; RETURN]; INT 2; APPLY; RETURN]; ACCESS 1; INT 4; EQ; IF; ENDLET & \{5\}   & \{\}\\

CLOSURE & CLOSURE [CLOSURE [ACCESS 1; INT 1; OP +; RETURN]; INT 2; APPLY; RETURN]; ACCESS 1; INT 4; EQ; IF; ENDLET & \{5\}   & \{[INT 1; RETURN]\{5\}\}\\

CLOSURE & ACCESS 1; INT 4; EQ; IF; ENDLET & \{5\} & \{[CLOSURE [ACCESS 1; INT 1; OP +; RETURN]\{5\}; INT 2; APPLY; RETURN]; [INT 1; RETURN]\{5\}\}\\

ACCESS & INT 4; EQ; IF; ENDLET & \{5\}   & \{5; [CLOSURE [ACCESS 1; INT 1; OP +; RETURN]\{5\}; INT 2; APPLY; RETURN]; [INT 1; RETURN]\{5\}\}\\

INT & EQ; IF; ENDLET & \{5\}   & \{4; 5; [CLOSURE [ACCESS 1; INT 1; OP +; RETURN]\{5\}; INT 2; APPLY; RETURN]; [INT 1; RETURN]\{5\}\}\\

EQ & IF; ENDLET & \{5\}   & \{false; [CLOSURE [ACCESS 1; INT 1; OP +; RETURN]\{5\}; INT 2; APPLY; RETURN]; [INT 1; RETURN]\{5\}\}\\

IF & CLOSURE [ACCESS 1; INT 1; OP +; RETURN]; INT 2; APPLY; RETURN & \{5\} & \{[ENDLET]; \{5\}\}\\

CLOSURE & INT 2; APPLY; RETURN & \{5\} & \{[ACCESS 1; INT 1; OP +; RETURN]\{5\}; [ENDLET]; \{5\}\}\\

INT & APPLY; RETURN & \{5\} & \{2; [ACCESS 1; INT 1; OP +; RETURN]\{5\}; [ENDLET]; \{5\}\}\\

APPLY & ACCESS 1; INT 1; OP +; RETURN & \{2; 5\} & \{[RETURN]; \{5\}; [ENDLET]; \{5\}\}\\

ACCESS & INT 1; OP +; RETURN & \{2; 5\} & \{2; [RETURN]; \{5\}; [ENDLET]; \{5\}\}\\

INT & OP +; RETURN & \{2; 5\} & \{1; 2; [RETURN]; \{5\}; [ENDLET]; \{5\}\}\\

OP & RETURN & \{2; 5\} & \{3; [RETURN]; \{5\}; [ENDLET]; \{5\}\}\\

RETURN & EMPTY & \{2; 5\} & \{3; [RETURN]; \{5\}; [ENDLET]; \{5\}\}\\

RETURN & EMPTY & \{5\} & \{3; [ENDLET]; \{5\}\}\\

ENDLET & EMPTY & \{\} & \{3\}\\
EMPTY
\end{tabular}
}

\newpage

{\small

\noindent Decompilation under $\mathcal{D}$ of a program compiled under $\mathcal{C}$, with an empty stack to begin, since the program is unexecuted.

\bigskip

\noindent $\mathcal{D}$(INT 5; LET x; CLOSURE [INT 1; RETURN]; CLOSURE [CLOSURE [ACCESS (x, 1); INT 1; OP +; RETURN]; INT 2; APPLY; RETURN]; ACCESS (x, 1); INT 4; EQ; IF; ENDLET, \{\})

\smallskip
\noindent Rule $\mathcal{D}$-CONST
\smallskip

\noindent $\mathcal{D}$(LET x; CLOSURE [INT 1; RETURN]; CLOSURE [CLOSURE [ACCESS (x, 1); INT 1; OP +; RETURN]; INT 2; APPLY; RETURN]; ACCESS (x, 1); INT 4; EQ; IF; ENDLET, \{\texttt{Int 5}\})

\smallskip
\noindent Rule $\mathcal{D}$-LET
\smallskip

\noindent \texttt{Let(x, }\texttt{Int 5}, $\mathcal{D}$(CLOSURE [INT 1; RETURN]; CLOSURE [CLOSURE [ACCESS (x, 1); INT 1; OP +; RETURN]; INT 2; APPLY; RETURN]; ACCESS (x, 1); INT 4; EQ; IF; ENDLET, \{\})\texttt{)}

\smallskip
\noindent Rule $\mathcal{D}$-CLOSURE
\smallskip

\noindent \texttt{Let(x, }\texttt{Int 5}, $\mathcal{D}$(CLOSURE [CLOSURE [ACCESS (x, 1); INT 1; OP +; RETURN]; INT 2; APPLY; RETURN]; ACCESS (x, 1); INT 4; EQ; IF; ENDLET, \{CLOSURE [INT 1; RETURN]\})\texttt{)}

\smallskip
\noindent Rule $\mathcal{D}$-CLOSURE
\smallskip

\noindent \texttt{Let(x, }\texttt{Int 5}, $\mathcal{D}$(ACCESS (x, 1); INT 4; EQ; IF; ENDLET, \{ CLOSURE [ACCESS (x, 1); INT 1; OP +; RETURN; INT 2; APPLY; RETURN]; CLOSURE [INT 1; RETURN]\})\texttt{)}

\smallskip
\noindent Rule $\mathcal{D}$-ACCESS
\smallskip

\noindent \texttt{Let(x, }\texttt{Int 5}, $\mathcal{D}$(INT 4; EQ; IF; ENDLET, \{\texttt{VarAccess(x, 1)}; CLOSURE [ACCESS (x, 1); INT 1; OP +; RETURN; INT 2; APPLY; RETURN]; CLOSURE [INT 1; RETURN]\})\texttt{)}

\smallskip
\noindent Rule $\mathcal{D}$-INT
\smallskip

\noindent \texttt{Let(x, }\texttt{Int 5}, $\mathcal{D}$(EQ; IF; ENDLET, \{\texttt{Int 4}; \texttt{VarAccess(x, 1)}; CLOSURE [ACCESS (x, 1); INT 1; OP +; RETURN]; INT 2; APPLY; RETURN; CLOSURE [INT 1; RETURN]\})\texttt{)}

\smallskip
\noindent Rule $\mathcal{D}$-EQ
\smallskip

\noindent \texttt{Let(x, }\texttt{Int 5}, $\mathcal{D}$(IF; ENDLET, \{\texttt{Eq(VarAccess (x, 1), Int 4)};  CLOSURE [ACCESS (x, 1); INT 1; OP +; RETURN]; INT 2; APPLY; RETURN; CLOSURE [INT 1; RETURN]\})\texttt{)}

\smallskip
\noindent Rule $\mathcal{D}$-IF
\smallskip

\noindent \texttt{Let(x, }\texttt{Int 5}, $\mathcal{D}$(ENDLET, \{\texttt{If(Eq(VarAccess (x, 1), Int 4)}, $\mathcal{D}$(CLOSURE [ACCESS (x, 1); INT 1; OP +; RETURN]; INT 2; APPLY; RETURN, \{\}), $\mathcal{D}$(CLOSURE [INT 1; RETURN], \{\})\texttt{)}


\smallskip
\noindent Rule $\mathcal{D}$-ENDLET
\smallskip

\noindent \texttt{Let(x, Int 5, If(Eq(VarAccess (x, 1), Int 4)}, $\mathcal{D}$(CLOSURE [ACCESS (x, 1); INT 1; OP +; RETURN]; INT 2; APPLY; RETURN, \{\}), $\mathcal{D}$(CLOSURE [INT 1; RETURN], \{\})\texttt{)}

\smallskip
\noindent Rule $\mathcal{D}$-CLOSURE then $\mathcal{D}$-INT then $\mathcal{D}$-RETURN
\smallskip

\noindent \texttt{Let(x, Int 5, If(Eq(VarAccess (x, 1), Int 4)}, $\mathcal{D}$(CLOSURE [ACCESS (x, 1); INT 1; OP +; RETURN]; INT 2; APPLY; RETURN, \{\}), \texttt{Int 1})\texttt{)}

\smallskip
\noindent Rule $\mathcal{D}$-CLOSURE
\smallskip

\noindent \texttt{Let(x, Int 5, If(Eq(VarAccess (x, 1), Int 4)}, $\mathcal{D}$(INT 2; APPLY; RETURN, \{CLOSURE [ACCESS (x, 1); INT 1; OP +; RETURN]\}), \texttt{Int 1})\texttt{)}

\smallskip
\noindent Rule $\mathcal{D}$-INT
\smallskip

\noindent \texttt{Let(x, Int 5, If(Eq(VarAccess (x, 1), Int 4)}, $\mathcal{D}$(APPLY; RETURN, \{\texttt{Int 2}; CLOSURE [ACCESS (x, 1); INT 1; OP +; RETURN]\}), \texttt{Int 1})\texttt{)}

\smallskip
\noindent Rule $\mathcal{D}$-APPLY
\smallskip

\noindent \texttt{Let(x, Int 5, If(Eq(VarAccess (x, 1), Int 4)}, \ \texttt{Apply(}$\mathcal{D}$(ACCESS (x, 1); INT 1; OP +; RETURN, \{\}), \texttt{Int 2)}, \texttt{Int 1})\texttt{)}

\smallskip
\noindent Rule $\mathcal{D}$-ACCESS then $\mathcal{D}$-INT then $\mathcal{D}$-OP then $\mathcal{D}$-RETURN
\smallskip

\noindent \texttt{Let(x, Int 5, If(Eq(VarAccess (x, 1), Int 4)}, \ \texttt{Apply(Lambda(Op(Var 1, Add, Int 1)), \texttt{Int 2)}, \texttt{Int 1})\texttt{)}}

}

\newpage

{\small

\noindent Decompilation under $\mathcal{D}$ of a program compiled under $\mathcal{C}$ and partially evaluated with $\mathcal{D}$.

\bigskip

\noindent $\mathcal{D}$(IF; ENDLET, {false; [CLOSURE [ACCESS(x, 1); INT 1; OP +; RETURN]; INT 2; APPLY; RETURN]; [INT 1; RETURN]})

\smallskip
\noindent Rule $\mathcal{D}$-IF
\smallskip

\noindent $\mathcal{D}$(ENDLET, \texttt{If (false,} $\mathcal{D}$([INT 1; RETURN], \{\})) $\mathcal{D}$([CLOSURE [ACCESS(x, 1); INT 1; OP +: RETURN]; INT 2; APPLY; RETURN], \{\}), \{\})

\smallskip
\noindent Rule $\mathcal{D}$-ENDLET
\smallskip

\noindent \texttt{If (Bool false,} $\mathcal{D}$([INT 1; RETURN], \{\}), $\mathcal{D}$([CLOSURE [ACCESS(x, 1); INT 1; OP +: RETURN]; INT 2; APPLY; RETURN]\{\}), \texttt{)}

\smallskip
\noindent Rule $\mathcal{D}$-INT then $\mathcal{D}$-RETURN
\smallskip


\noindent \texttt{If (Bool false,} \texttt{Int 1}, $\mathcal{D}$([CLOSURE [ACCESS(x, 1); INT 1; OP +: RETURN]; INT 2; APPLY; RETURN]\{\}),\texttt{)}

\smallskip
\noindent Rule $\mathcal{D}$-CLOSURE
\smallskip


\noindent \texttt{If (Bool false,} \texttt{Int 1}, $\mathcal{D}$([INT 2; APPLY; RETURN], \{[ACCESS(x, 1); INT 1; OP +: RETURN]\})\texttt{)}

\smallskip
\noindent Rule $\mathcal{D}$-INT
\smallskip

\noindent \texttt{If (Bool false,} \texttt{Int 1}, $\mathcal{D}$([APPLY; RETURN], \{\texttt{Int 2}; [ACCESS(x, 1); INT 1; OP +: RETURN]\})\texttt{)}

\smallskip
\noindent Rule $\mathcal{D}$-APPLY
\smallskip

\noindent \texttt{If (Bool false,}, \texttt{Int 1}, \texttt{Apply (Lambda(x, }$\mathcal{D}$(ACCESS(x, 1); INT 1; OP +: RETURN, \{\}), \texttt{Int 2))}\texttt{)}

\smallskip
\noindent Rule $\mathcal{D}$-ACCESS
\smallskip

\noindent \texttt{If (Bool false,} \texttt{Int 1}, \texttt{Apply (Lambda(x, }$\mathcal{D}$(INT 1; OP +; RETURN, \{\texttt{VarAccess(x, 1)}\}), \texttt{Int 2))}\texttt{)}

\smallskip
\noindent Rule $\mathcal{D}$-INT
\smallskip

\noindent \texttt{If (Bool false,} \texttt{Int 1}, \texttt{Apply (Lambda(x, }$\mathcal{D}$(OP +; RETURN, \{\texttt{Int 1; VarAccess(x, 1)}\}), \texttt{Int 2))}\texttt{)}

\smallskip
\noindent Rule $\mathcal{D}$-OP
\smallskip

\noindent \texttt{If (Bool false,} \texttt{Int 1}, \texttt{Apply (Lambda(x, }$\mathcal{D}$(RETURN, \{\texttt{Op(VarAccess(x, 1), Add, Int 1)}\}), \texttt{Int 2))}\texttt{)}

\smallskip
\noindent Rule $\mathcal{D}$-RETURN
\smallskip

\noindent \texttt{If (Bool false,} \texttt{Int 1}, \texttt{Apply (Lambda(x, Op(VarAccess(x, 1), Add, Int 1)}, \texttt{Int 2))}\texttt{)}

\bigskip

\noindent This is the program \texttt{if false then 1 else (fun x -> x + 1) 2}, as required.

}\end{landscape}

\end{document}
