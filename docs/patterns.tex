\documentclass[10pt]{article}
\usepackage{nameref}
\usepackage{cleveref}
\usepackage{palatino}
\usepackage[scaled=0.9]{beramono}
\usepackage[T1]{fontenc}
\usepackage[protrusion=true,expansion=true]{microtype}
\usepackage[draft=false]{hyperref}
\usepackage[left=4cm, right=4cm, top=2cm, bottom=1.8cm]{geometry}
\usepackage{amsmath}
\usepackage{amssymb}
\usepackage{ulem}
\usepackage{attrib}
\begin{document}

\title{Seaching through a Trace with Patterns}
\maketitle

The principal method by which a bug will be located in our system is by the (semi-)automatic edition of the program trace. This can happen as the program is evaluated, or by interactive searching afterward. But we need a way to express the searches.

Text-based approaches (e.g regular expressions) are not much use. Fine for searching for a function name, but no good for tree-based data like functional programs. We want to be able to say, for example:

\begin{enumerate}
\item ``Find any function application in the trace taking an empty tree as input.''
\item ``Show me all calls to function \texttt{f}''
\item ``Show me any time a list begins with a negative number''
\end{enumerate}

\noindent Patterns for 1, 2, 3 might be ``\texttt{\_ Lf}'', ``f \_'', and ``[-\_; ...]''.

\section{Patterns}

For example...

\begin{tabular}{ll}
\texttt{\_} & wildcard\\
\texttt{()} & parentheses\\
\texttt{{[}p; p{]}} & lists (similarly for records, tuples etc)\\
\texttt{...} & to indicate the tail of a list\\
\texttt{text} & literal text\\
\texttt{remove*} & wildcards as part of text\\
\end{tabular}

\begin{itemize}
\item We use the same lexical conventions as OCaml, so we can reuse the OCaml lexer. The parser should not be too difficult, just must have a subset of the associativities / precedences of the OCaml one.
\item The parser should accept \textit{any} string, just counts as text to match.
\end{itemize}

\section{TODO}

\begin{enumerate}
\item Define a small example pattern language
\item Choose a subset of \texttt{Tinyocaml.t}
\item Find Example program, trace, and patterns
\item Write the pattern parser
\item Write the matcher, which sees if a pattern matches a line of the trace, by matching the tree of the pattern against the tree of that line, rather than text against text. Need a way to indicate the matched part, say by underlining.
\end{enumerate}

\section{Difficult examples}

How do we match pattern \texttt{1 + 2 + 3}. Does the tree \texttt{1 + (2 + 3)} match? Could we just remove all parentheses? Too many false matches?

\section{The text approach}

Advantages:
\begin{itemize}
\item Probably simpler to implement
\item Regular expressions are familiar to the user - no need for our own pattern language, or at least we could base it on something.
\end{itemize}

\noindent Limitations:

\begin{itemize}
\item No clean support for dealing with parentheses, associativity, precedence.
\item We want wildcards which correspond to nodes of a tree, not text items!
\end{itemize}

\section{The tree approach}

Advantages:
\begin{itemize}
\item Mirrors the structure of the language - feels a bit like pattern matching -- OCaml users already familiar -- in fact, will be a strict superset of OCaml pattern matching.
\end{itemize}

\noindent Limitations:
\begin{itemize}
\item Dealing with precedence and associativity means the tree shapes don't necessarily match.
\end{itemize}

\section{References}

\begin{enumerate}
\item ``Pattern Matching in Trees'' (Hoffmann, O'Donnell, 1982). This early paper considers the pattern matching of trees in terms of interpreters for languages whose evaluation is defined by term rewriting rules. This is relevant to our wider work, despite finding it in this context.
\item ``Pattern matching syntax trees''\\ \url{http://checkmyworking.com/2014/06/pattern-matching-syntax-trees/} This is interesting work in the context of matching up childrens' answers to computer algebra questions with the expected answers. It uses a special kind of pattern matching which knows about nesting of parentheses, and commutivity (for example, $x + 1$ and $1 + x$ are considered equal).
\item There may be good research in this area for XML processing systems -- it's all trees. XPath. See \url{https://theantlrguy.atlassian.net/wiki/display/~admin/2013/09/01/Matching+parse+tree+patterns,+paths}
\item \textit{Darren C. Atkinson and William G. Griswold, "Effective pattern matching of source code using abstract syntax patterns} look this one up. 
\item Darren C. Atkinson and William Griswold and Collin McCurdy \textit{Fast, Flexible Syntactic Pattern Matching and Processing}.
\item A toolset for FORTRAN programs -- contains pattern matching over fortran statements \url{https://www.irisa.fr/caps/projects/TSF/demo/doc/draft.html}
\item Roger F. Crew, \textit{ASTLOG: A Language for Examining Abstract Syntax Trees}.
\end{enumerate}

\section{Notes on First Implementation of on-line search}

An initial implementation, just with regular expressions for now, has been written to explore what kind of tools are needed for searching program traces.

\begin{enumerate}
\item The option \texttt{-search <searchterm>}, given a regular expression, shows only those lines matching the search term.
\item The option \texttt{-n <n>} prints only up to $n$ matches.
\item The option \texttt{-invert-search} which shows those lines not matching the search term.
\item The option \texttt{-after <searchterm>} which further restricts the output to lines occurring after and including the first to match the given searchterm, which must match something which matches the \texttt{-search} searchterm.
\item The option \texttt{-until <searchterm>} which further restricts the output to lines occurring after and including the first to match the given searchterm, which must match something which matches the \texttt{-search} searchterm. 
\item Options \texttt{-after-any <searchterm>} and \texttt{-until-any <searchterm>} which are like \texttt{-after} and \texttt{-until} but which match on any line, not just those matching the main \texttt{-search} search term. Alternative to not in addition to \texttt{-until}/\texttt{-after}.
\item Options \texttt{-invert-after} and \texttt{-invert-until} which invert the meanings of \texttt{-after}, \texttt{-after-any}, \texttt{-until} and \texttt{-until-any} just like \texttt{-invert-search}.
\item The option \texttt{-around <n>} which prints $n$ lines (not $n$ search results) either side of the printed one.
\item An option \texttt{-stop} to stop computation after search result(s) printed. Defaults is to carry on to the end.
\end{enumerate}

\section{Offline vs Online search: Pros and cons}

\end{document}
















