\documentclass[10pt]{article}
\usepackage{nameref}
\usepackage{cleveref}
\usepackage{palatino}
\usepackage[scaled=0.9]{beramono}
\usepackage[T1]{fontenc}
\usepackage[protrusion=true,expansion=true]{microtype}
\usepackage[draft=false]{hyperref}
\usepackage[left=4cm, right=4cm, top=2cm, bottom=1.8cm]{geometry}
\usepackage{amsmath}
\usepackage{amssymb}
\usepackage{ulem}
\usepackage{attrib}
\begin{document}

\title{Seaching through a Trace with Patterns}
\maketitle

The principal method by which a bug will be located in our system is by the (semi-)automatic edition of the program trace. This can happen as the program is evaluated, or by interactive searching afterward. But we need a way to express the searches.

Text-based approaches (e.g regular expressions) are not much use. Fine for searching for a function name, but no good for tree-based data like functional programs. We want to be able to say, for example:

\begin{enumerate}
\item ``Find any function application in the trace taking an empty tree as input.''
\item ``Show me all calls to function \texttt{f}''
\item ``Show me any time a list begins with a negative number''
\end{enumerate}

\noindent Patterns for 1, 2, 3 might be ``\texttt{\_ Lf}'', ``f \_'', and ``[-\_; ...]''.

\section{Patterns}

For example...

\begin{tabular}{ll}
\texttt{\_} & wildcard\\
\texttt{()} & parentheses\\
\texttt{{[}p; p{]}} & lists (similarly for records, tuples etc)\\
\texttt{...} & to indicate the tail of a list\\
\texttt{text} & literal text\\
\texttt{remove*} & wildcards as part of text\\
\end{tabular}

\begin{itemize}
\item We use the same lexical conventions as OCaml, so we can reuse the OCaml lexer. The parser should not be too difficult, just must have a subset of the associativities / precedences of the OCaml one.
\item The parser should accept \textit{any} string, just counts as text to match.
\end{itemize}

\section{TODO}

\begin{enumerate}
\item Define a small example pattern language
\item Choose a subset of \texttt{Tinyocaml.t}
\item Find Example program, trace, and patterns
\item Write the pattern parser
\item Write the matcher, which sees if a pattern matches a line of the trace, by matching the tree of the pattern against the tree of that line, rather than text against text. Need a way to indicate the matched part, say by underlining.
\end{enumerate}

\end{document}
















